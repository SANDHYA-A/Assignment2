\documentclass[journal,12pt,twocolumn]{IEEEtran}
\IEEEoverridecommandlockouts
\usepackage{cite}
\usepackage{amsmath,amssymb,amsfonts,bm}
\usepackage{mathtools}
\let\vec\mathbf
\newcommand{\myvec}[1]{\ensuremath{\begin{pmatrix}#1\end{pmatrix}}}
\newcommand{\norm}[1]{\left\lVert#1\right\rVert}

\begin{document}

\title{Matrix Theory EE5609 - Assignment 2\\
Find Inverse of a matrix using Elementary transformations
}

\author{\IEEEauthorblockN{Sandhya Addetla}\\
\IEEEauthorblockA{PhD Artificial Inteligence Department} \\
15-Sep-2020\\
AI20RESCH14001\\
 }

\maketitle
\begin{abstract}
This document provides a solution for finding inverse of a matrix using elementary transformations.
\end{abstract}

\section{Problem Statement}
Using elementary transformations, find the inverse of the matrix  A = $\myvec{2 &0 &-1\\5 &1 &0\\0 &1 &3} $, if it exists.

\section{Theory}
For any n $\times$  n matrix A, if the augmented matrix $ [A | I]$ is transformed into a matrix of the form$ [I|B]$, then the matrix A is invertible and the inverse matrix A$^{-1}$ is given by B. If the reduced row echelon form matrix for$ [A | I]$ is not of the form $[ I | B]$, then the matrix A is not invertible.

\section{Solution}

The augmented matrix $ [A | I]$ is as given below:- 

\begin{align}
\myvec{
		 2 &0 &-1&  1 & 0 &  0 \\
		   5 &1 &0&  0 &  1 &  0 \\
		 0 &1 &3 &  0 &  0 &  1 \\
		 }
 \end{align}
 We apply the elementary row operations on $ [A | I]$ as follows :-
 
\begin{align}
 [A | I] = \myvec{
		 2 &0 &-1&  1 & 0 &  0 \\
		   5 &1 &0&  0 &  1 &  0 \\
		 0 &1 &3 &  0 &  0 &  1 \\
		 } \\
\xrightarrow{R_2\rightarrow 2R_2 - 5R_1} \myvec{
		 2 &0 &-1&  1 & 0 &  0 \\
		  0 &2 &5&  -5 &  2 &  0 \\
		 0 &1 &3 &  0 &  0 &  1 \\
		 } 
		 \\
 \xrightarrow{R_3\rightarrow 2R_3 - R_2} \myvec{
		 2 &0 &-1&  1 & 0 &  0 \\
		  0 &2 &5&  -5 &  2 &  0 \\
		 0 &0 &1 &  5 &  -2 &  2 \\
		 } \\
 \xrightarrow{\substack{R_1\rightarrow \frac{R_1}{2}\\R_2\rightarrow \frac{R_2}{2}}} \myvec{
		 1 & 0 & \frac{-1}{2} & \frac{-1}{2} & 0 &  0 \\
		  0 & 1 & \frac{5}{2} & \frac{-5}{2} &  1 &  0 \\
		 0 & 0 &1 &  5 &  -2 &  2 \\
		 } 	\\	 
\xrightarrow{\substack{R_1\rightarrow R_1  + \frac{R_3}{2}\\R_2\rightarrow R_2 - \frac{5}{2}R_3}} \myvec{
		 1 & 0 & 0 & 3 & -1 &  1 \\
		  0 & 1 & 0 & -15 &  6 &  -5 \\
		 0 & 0 &1 &  5 &  -2 &  2 \\
		 } 	\\	
\end{align}
\section{Conclusion}
By performing elementary transormations on augmented matrix$ [A | I]$ , we obtained the augmented matrix in the form $ [I|B]$. 
Hence we can conclude that the matrix A is invertible and inverse of the matrix is:-
\begin{align}
A^{-1} = \myvec{
		  3 & -1 &  1 \\
		 -15 &  6 & -5 \\
		  5 &  -2 &  2 \\
		 }
\end{align}

\end{document}